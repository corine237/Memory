\chapter*{RESUME}

	Exerçant dans le domaine de l'informatique et travaillant au sein d'une entreprise qui se veut promouvoir celui-ci dans l'éducation, nous avons trouvé intéressant de se lancer dans l'amélioration de l'apprentissage des mathématiques qui servira de soutient en milieu scolaire. Matière dont le mythe sur la difficulté sevit toujours. Après une recherche empirique, le mythe s'avère fondé car nous constatons que les élèves éprouvent de réelles difficultés en cette matière. Le prouve également la faible orientation de ceux-ci vers les filières scientifiques. Cette difficulté causée la plupart du temps pas l'accumulation des lacunes, nous nous sommes demandés Comment recommander les cours à chaque apprenant en fonction de ses difficultés spécifiques ? Ce mémoire présente AREDU, un système de recommandation de cours basé sur la factorisation des matrices, méthode du filtrage collaboratif dont la technique est le décomposition à valeurs singulières(SVD). En partant des données réelles collectées au sein de 3 établissements dans la ville de Douala, nous avons entrainé notre modèle de telle que sorte à recommander des cours à l'apprenant selon sa difficulté suite à un exercice ou évaluation traité. Les résultats de l'expérience se sont avérés concluant selon la satisfaction des feedbacks reçus, ce qui appuie le rajout de nouveaux aspects proposés. \\
	
	\textbf{Mots clés:} Système de recomandation, filtrage collaboratif, factorisation des matrices, SVD, recommandation de cours.