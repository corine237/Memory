\chapter*{INTRODUCTION}

	L’augmentation de la capacité d’accueil des établissements et la répartitions des programmes de matières dans un délai engendre une complexification de la gestion des élèves. Le suivi des chaque élève s’avère un peu plus difficile dans la pratique de part les méthodologies pédagogiques adoptées et le temps imparti. Bien que l’intelligence artificielle suscite un intéret particulier dans le domaine de l’éducation, il est necessaire de se demander, comment recommander les cours à chaque apprenant en fonction de ses difficultés spécifiques. Cette dernière interrogation semble pertinente et résonne comme un appel urgent à mettre en place des outils d’analyse et d’accompagnement optimale afin de proposer un système de recommandation qui identifiera les besoins de l’apprenant et lui proposera des leçons adaptées tout en consolidant les prérequis. C’est fort de cette idée de solution que nous avons choisi de répondre à la problématique énoncée sous la thématique suivante: ETUDE ET REALISATION D’UN SYSTEME DE RECOMMANDATION POUR L’ APPRENTISSAGE DES MATHÉMATIQUES EN CLASSE DE 6e AU CAMEROUN. Ce qui nous confère comme objectif, proposer un système de recommandation qui identifie les besoins de l’apprenant et lui propose des leçons adaptées. Afin de traiter ce sujet, nous avons établi un plan de recherche qui consiste dans un premier temps à effectuer une série de tests sur un échantillon d’élèves dans trois lycées de la ville de Douala. Dans un second temps, nous avons mené des entretiens semi-directifs avec des enseignants des collèges et lycées d’enseignement secondaires ainsi qu’avec des acteurs du système éducatif au Cameroun. La recherche empirique a été complétée par de nombreuses lectures sur le sujet.
Le présent mémoire qui tient lieu de la restitution du déroulement de cette expérience cruciale va s’articuler autour de 04 chapitres répartis en 2 parties. La première partie, portant sur l’état de l’art, se divise en deux chapitres notamment la revue de littérature qui traite de l’intelligence artificielle en milieu scolaire et de l’apprentissage adaptatif  au premier chapitre et de la présentation du projet en deuxième chapitre. La deuxième partie intitulée mise en œuvre du système, qui elle aussi comporte deux chapitres parmi lesquels dont premier présente l’ensemble de notre dossier d’analyse et conception en tant que troisième chapitre, et le second, la réalisation du projet et ses résultats (quatrième chapitre). Le plan ainsi balisé servira d’ossature articulaire de notre présentation. 
